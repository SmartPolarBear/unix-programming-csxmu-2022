\documentclass[a4paper,twoside]{article}
\usepackage{blindtext}  
\usepackage{geometry}

% Chinese support
\usepackage[UTF8, scheme = plain]{ctex}

% Page margin layout
\geometry{left=2.3cm,right=2cm,top=2.5cm,bottom=2.0cm}


\usepackage{listings}
\usepackage{xcolor}
\usepackage{geometry}
\usepackage{amsmath}
\usepackage{float}
\usepackage{hyperref}

\usepackage{graphics}
\usepackage{graphicx}
\usepackage{epsfig}
\usepackage{float}

\usepackage{algorithm}
\usepackage[noend]{algpseudocode}

\usepackage{booktabs}
\usepackage{threeparttable}
\usepackage{longtable}
\usepackage{listings}
\usepackage{tikz}
\usepackage{multicol}

\usepackage{caption}
\usepackage{subcaption}

% cite package, to clean up citations in the main text. Do not remove.
\usepackage{cite}

\usepackage{color,xcolor}

%% The amssymb package provides various useful mathematical symbols
\usepackage{amssymb}
%% The amsthm package provides extended theorem environments
\usepackage{amsthm}
\usepackage{amsfonts}
\usepackage{enumerate}
\usepackage{enumitem}
\usepackage{listings}

\usepackage{textcomp}

\usepackage{indentfirst}
\setlength{\parindent}{2em} % Make two letter space in the first paragraph
\usepackage{setspace}
\linespread{1.5} % Line spacing setting
\usepackage{siunitx}
\setlength{\parskip}{0.5em} % Paragraph spacing setting

% \usepackage[contents =22920202204622, scale = 10, color = black, angle = 50, opacity = .10]{background}

\renewcommand{\figurename}{图}
\renewcommand{\lstlistingname}{代码} 
\renewcommand{\tablename}{表格}
\renewcommand{\contentsname}{目录}
\floatname{algorithm}{算法}

\graphicspath{ {images/} }

%%%%%%%%%%%%%
\newcommand{\StudentNumber}{22920202204622}  % Fill your student number here
\newcommand{\StudentName}{熊恪峥}  % Replace your name here
\newcommand{\PaperTitle}{实验(三) 编写程序myfind}  % Change your paper title here
\newcommand{\PaperType}{Unix程序设计} % Replace the type of your report here
\newcommand{\Date}{2022年10月17日}
\newcommand{\College}{信息学院}
\newcommand{\CourseName}{Unix程序设计}
%%%%%%%%%%%%%

%% Page header and footer setting
\usepackage{fancyhdr}
\usepackage{lastpage}
\pagestyle{fancy}
\fancyhf{}
% This requires the document to be twoside
\fancyhead[LO]{\texttt{\StudentName }}
\fancyhead[LE]{\texttt{\StudentNumber}}
\fancyhead[C]{\texttt{\PaperTitle }}
\fancyhead[R]{\texttt{第{\thepage}页,共\pageref*{LastPage}页}}


\title{\PaperTitle}
\author{\StudentName}
\date{\Date}

\lstset{
	basicstyle          =   \sffamily,          % 基本代码风格
	keywordstyle        =   \bfseries,          % 关键字风格
	commentstyle        =   \rmfamily\itshape,  % 注释的风格,斜体
	stringstyle         =   \ttfamily,  % 字符串风格
	flexiblecolumns,                % 别问为什么,加上这个
	numbers             =   left,   % 行号的位置在左边
	showspaces          =   false,  % 是否显示空格,显示了有点乱,所以不现实了
	numberstyle         =   \zihao{-5}\ttfamily,    % 行号的样式,小五号,tt等宽字体
	showstringspaces    =   false,
	captionpos          =   t,      % 这段代码的名字所呈现的位置,t指的是top上面
	frame               =   lrtb,   % 显示边框
}

\lstdefinestyle{PythonStyle}{
	language        =   Python, % 语言选Python
	basicstyle      =   \zihao{-5}\ttfamily,
	numberstyle     =   \zihao{-5}\ttfamily,
	keywordstyle    =   \color{blue},
	keywordstyle    =   [2] \color{teal},
	stringstyle     =   \color{magenta},
	commentstyle    =   \color{red}\ttfamily,
	breaklines      =   true,   % 自动换行,建议不要写太长的行
	columns         =   fixed,  % 如果不加这一句,字间距就不固定,很丑,必须加
	basewidth       =   0.5em,
}

\definecolor{keycolor}{RGB}{172, 42, 42}
\definecolor{mbleu}{RGB}{64,96,127}
\definecolor{vimvert}{RGB}{46, 139, 87}

\lstdefinestyle{MakefileBaseStyle}{
basicstyle=\ttfamily\scriptsize\color{black!90},%
stringstyle=\itshape\color{magenta},%
showstringspaces=false,%
keywordstyle=\bfseries\color{keycolor},%
commentstyle=\color{blue}\slshape,%
framexleftmargin=1mm,%
backgroundcolor=\color{black!2},%
}

\lstdefinestyle{MakefileStyle}{
	otherkeywords={.SUFFIXES},
	morekeywords={SUFFIX, CPP_,},
	moredelim=[is][\color{mbleu}]{/*}{*/},
	style=MakefileBaseStyle,%
	morecomment=[l][commentstyle]{\#},%
	emphstyle={\color{vimvert}},%
	moredelim=[s][\color{vimvert}]{\$(}{)}%
}

\lstdefinestyle{CppStyle}{
	language        =   c++,
	basicstyle      =   \zihao{-5}\ttfamily,
	numberstyle     =   \zihao{-5}\ttfamily,
	keywordstyle    =   \color{blue},
	keywordstyle    =   [2] \color{teal},
	stringstyle     =   \color{magenta},
	commentstyle    =   \color{red}\ttfamily,
	breaklines      =   true,   % 自动换行,建议不要写太长的行
	columns         =   fixed,  % 如果不加这一句,字间距就不固定,很丑,必须加
	basewidth       =   0.5em,
}

\algnewcommand\algorithmicinput{\textbf{Input:}}
\algnewcommand\algorithmicoutput{\textbf{Output:}}
\algnewcommand\Input{\item[\algorithmicinput]}%
\algnewcommand\Output{\item[\algorithmicoutput]}%

\usetikzlibrary{positioning, shapes.geometric}

% 流程图定义基本形状
\tikzstyle{startstop} = [rectangle, rounded corners, minimum width = 2cm, minimum height=1cm,text centered, draw = black]
\tikzstyle{io} = [trapezium, trapezium left angle=70, trapezium right angle=110, minimum width=2cm, minimum height=1cm, text centered, draw=black]
\tikzstyle{process} = [rectangle, minimum width=3cm, minimum height=1cm, text centered, draw=black]
\tikzstyle{decision} = [diamond, aspect = 3, text centered, draw=black]
% 箭头形式
\tikzstyle{arrow} = [->,>=stealth]

\newtheorem{assumption}{Assumption}[section]

\begin{document}
	
%%%%%%%%%%%%%%%%%%%%%%%%%%%%%%%%%%%%%%%%%%%%
\makeatletter % change default title style
\renewcommand*\maketitle{%
	\begin{center} 
		\bfseries  % title 
		{\LARGE \@title \par}  % LARGE typesetting
		\vskip 1em  %  margin 1em
		{\global\let\author\@empty}  % no author information
		{\global\let\date\@empty}  % no date
		\thispagestyle{empty}   %  empty page style
	\end{center}%
	\setcounter{footnote}{0}%
}
\makeatother
%%%%%%%%%%%%%%%%%%%%%%%%%%%%%%%%%%%%%%%%%%%%
	
	
\thispagestyle{empty}

\vspace*{1cm}

\begin{figure}[h]
	\centering
	\includegraphics[width=4.0cm]{logo.png}
\end{figure}

\vspace*{1cm}

\begin{center}
	\Huge{\textbf{\PaperType}}
	
	\Large{\PaperTitle}
\end{center}

\vspace*{1cm}

\begin{table}[h]
	\centering	
	\begin{Large}
		\renewcommand{\arraystretch}{1.5}
		\begin{tabular}{p{3cm} p{5cm}<{\centering}}
			姓\qquad 名 & \StudentName  \\
			\hline
			学\qquad号 & \StudentNumber \\
			\hline
			日\qquad期 & \Date  \\
			\hline
			学\qquad院 & \College  \\
			\hline
			课程名称 & \CourseName  \\
			\hline
		\end{tabular}
	\end{Large}
\end{table}

\newpage

\title{
	\Large{\textcolor{black}{\PaperTitle}}
}
	
	
\maketitle
	
\tableofcontents
 
\newpage
\setcounter{page}{1}

\begin{spacing}{1.2}

\section{实验内容}

编写程序myfind

\subsection{命令语法}
myfind  \textlangle pathname\textrangle \ [-comp \textlangle filename\textrangle \ \textbar \ -name \textlangle str\textrangle  …]

\subsection{命令语义}

\begin{enumerate}
	\item myfind  \textlangle pathname\textrangle 的功能:除了具有与程序4-7相同的功能外,还要输出在\textlangle pathname\textrangle  目录子树之下,文件长度不大于4096字节的常规文件,在所有允许访问的普通文件中所占的百分比。程序不允许打印出任何路径名。
	\item myfind  \textlangle pathname\textrangle  \  -comp  \textlangle filename\textrangle  的功能:\textlangle filename\textrangle  是常规文件的路径名(非目录名,但是其路径可以包含目录)。命令仅仅输出在\textlangle pathname\textrangle  目录子树之下,所有与\textlangle filename\textrangle  文件内容一致的文件的绝对路径名。不允许输出任何其它的路径名,包括不可访问的路径名。
	\item myfind  \textlangle pathname\textrangle \ -name \textlangle str\textrangle  …的功能:\textlangle str\textrangle  …是一个以空格分隔的文件名序列(不带路径)。命令输出在\textlangle pathname\textrangle  目录子树之下,所有与\textlangle str\textrangle  …序列中文件名相同的文件的绝对路径名。不允许输出不可访问的或无关的路径名。
\end{enumerate}

\section{程序设计与实现}

\subsection{程序设计}

\subsection{程序实现}

\section{程序测试}

\section{问题与改进}

\appendix
\clearpage
\section*{附录:代码清单}
\addcontentsline{toc}{part}{附录:代码清单}
\label{sec:code}

\subsection{Makefile}

\begin{lstlisting}[numbers=left,style=MakefileStyle,caption=Makefile,label={code:makefile}]
OBJS = main.o \
	   error.o \
	   pathalloc.o \

CC = gcc
CFLAGS = -Wall -g -std=c99

%.o: %.c
	$(CC) $(CFLAGS) -c $< -o $@

myfind: $(OBJS)
	$(CC) $(CFLAGS) $(OBJS) -o myfind

all: myfind

clean:
	rm -f *.o myfind

.PHONY: clean all
\end{lstlisting}

\subsection{myfind.c}

\begin{lstlisting}[numbers=left,style=CppStyle,caption=程序实现,label={code:impl}]
#include "apue.h"
#include <dirent.h>
#include <limits.h>

#include <libgen.h> // basename


typedef int (Callback)(const char*, const struct stat*, int);

static Callback simple_statistic;
static Callback content_compare;
static Callback name_compare;

static int myftw(char*, Callback*);

static int dopath(Callback*);

static int compare_file(const char* file1, const char* file2);

static long nreg, ndir, nblk, nchr, nfifo, nslink, nsock, ntot, nless4k;

#define NARGS 16

static char* names[NARGS];
static int name_count = 0;

static char* comp_filename = NULL;
static struct stat comp_stat;

int main(int argc, char* argv[]) {
    int ret;
    if (!(argc == 2 || argc >= 4)) {
        err_quit("usage: myfind <params>");
    }

    if (argc == 2)
    {
        ret = myftw(argv[1], simple_statistic);
        ntot = nreg + ndir + nblk + nchr + nfifo + nslink + nsock;
        if (ntot == 0) {
            ntot = 1;
        }

        printf("regular files  = %7ld, %5.2f %%\n", nreg, nreg * 100.0 / ntot);
        printf("directories    = %7ld, %5.2f %%\n", ndir, ndir * 100.0 / ntot);
        printf("block special  = %7ld, %5.2f %%\n", nblk, nblk * 100.0 / ntot);
        printf("char special   = %7ld, %5.2f %%\n", nchr, nchr * 100.0 / ntot);
        printf("FIFOs          = %7ld, %5.2f %%\n", nfifo, nfifo * 100.0 / ntot);
        printf("symbolic links = %7ld, %5.2f %%\n", nslink, nslink * 100.0 / ntot);
        printf("sockets        = %7ld, %5.2f %%\n", nsock, nsock * 100.0 / ntot);
        printf("smaller than 4k= %7ld, %5.2f %%\n", nless4k, nless4k * 100.0 / ntot);

    }
    else if (argc >= 4)
    {
        char* pathname = argv[1];
        if (strcmp(argv[2], "-comp") == 0)
        {
            comp_filename = argv[3];
            if (lstat(comp_filename, &comp_stat) < 0)
            {
                err_sys("lstat error for %s", comp_filename);
            }
            ret = myftw(pathname, content_compare);
        }
        else if (strcmp(argv[2], "-name") == 0)
        {
            for (int i = 3; i < argc; i++)
            {
                names[i - 3] = argv[i];
                name_count++;
            }
            ret = myftw(pathname, name_compare);
        }
        else
        {
            err_quit("usage: myfind <params>");
        }
    }

    return ret;
}

#define FTW_F 1
#define FTW_D 2
#define FTW_DNR 3
#define FTW_NS 4

static char* fullpath;
static size_t pathlen;

static int myftw(char* pathname, Callback* func) {
    fullpath = path_alloc(&pathlen);

    if (pathlen <= strlen(pathname)) {
        pathlen = strlen(pathname) * 2;
        if ((fullpath = realloc(fullpath, pathlen)) == NULL) {
            err_sys("realloc failed");
        }
    }
    strcpy(fullpath, pathname);
    return (dopath(func));
}

static int dopath(Callback* func) {
    struct stat statbuf;
    struct dirent* dirp;
    DIR* dp;
    int ret, n;
    if (lstat(fullpath, &statbuf) < 0) {
        return (func(fullpath, &statbuf, FTW_NS));
    }
    if (S_ISDIR(statbuf.st_mode) == 0) {
        return (func(fullpath, &statbuf, FTW_F));
    }

    if ((ret = func(fullpath, &statbuf, FTW_D)) != 0) {
        return (ret);
    }
    n = strlen(fullpath);
    if (n + NAME_MAX + 2 > pathlen) {
        pathlen *= 2;
        if ((fullpath = realloc(fullpath, pathlen)) == NULL) {
            err_sys("realloc failed");
        }
    }
    fullpath[n++] = '/';
    fullpath[n] = 0;
    if ((dp = opendir(fullpath)) == NULL) {
        return (func(fullpath, &statbuf, FTW_DNR));
    }
    while ((dirp = readdir(dp)) != NULL) {
        if (strcmp(dirp->d_name, ".") == 0 || strcmp(dirp->d_name, "..") == 0) {
            continue;
        }
        strcpy(&fullpath[n], dirp->d_name);
        if ((ret = dopath(func)) != 0) {
            break;
        }
    }
    fullpath[n - 1] = 0;
    if (closedir(dp) < 0) {
        err_ret("can't close directory %s", fullpath);
    }
    return (ret);
}

static int simple_statistic(const char* pathname, const struct stat* statptr, int type) {
    switch (type) {
    case FTW_F:
        switch (statptr->st_mode & S_IFMT) {
        case S_IFREG:
            nreg++;
            if (statptr->st_size <= 4096) {
                nless4k++;
            }
            break;
        case S_IFBLK:
            nblk++;
            break;
        case S_IFCHR:
            nchr++;
            break;
        case S_IFIFO:
            nfifo++;
            break;
        case S_IFLNK:
            nslink++;
            break;
        case S_IFSOCK:
            nsock++;
            break;
        case S_IFDIR:
            err_dump("for S_IFDIR for %s", pathname);
        }
        break;
    case FTW_D:
        ndir++;
        break;
    case FTW_DNR:
        err_ret("can't read directory %s", pathname);
        break;
    case FTW_NS:
        err_ret("stat error for %s", pathname);
        break;
    default:
        err_dump("unknown type %d for pathname %s", type, pathname);
    }
    return 0;
}

static int content_compare(const char* pathname, const struct stat* statptr, int type) {
    switch (type) {
    case FTW_F:
        switch (statptr->st_mode & S_IFMT) {
        case S_IFREG:
        case S_IFBLK:
        case S_IFCHR:
        case S_IFIFO:
        case S_IFLNK:
        case S_IFSOCK:
            if (statptr->st_size > 0) {
                if (statptr->st_size == comp_stat.st_size &&
                    compare_file(pathname, comp_filename)) {

                    char* real = realpath(pathname, NULL);
                    printf("%s\n", real);
                    free(real);
                }
            }
            break;
        case S_IFDIR:
            err_dump("for S_IFDIR for %s", pathname);
        }
        break;
    case FTW_D:
        break;
    case FTW_DNR:
        err_ret("can't read directory %s", pathname);
        break;
    case FTW_NS:
        err_ret("stat error for %s", pathname);
        break;
    default:
        err_dump("unknown type %d for pathname %s", type, pathname);
    }
    return 0;
}

static int name_compare(const char* pathname, const struct stat* statptr, int type) {
    switch (type) {
    case FTW_F:
        switch (statptr->st_mode & S_IFMT) {
        case S_IFREG:
        case S_IFBLK:
        case S_IFCHR:
        case S_IFIFO:
        case S_IFLNK:
        case S_IFSOCK:
            for (int i = 0;i < name_count;i++)
            {
                if (strcmp(basename(pathname), names[i]) == 0)
                {
                    printf("%s\n", pathname);
                }
            }
            break;
        case S_IFDIR:
            err_dump("for S_IFDIR for %s", pathname);
        }
        break;
    case FTW_D:
        break;
    case FTW_DNR:
        err_ret("can't read directory %s", pathname);
        break;
    case FTW_NS:
        err_ret("stat error for %s", pathname);
        break;
    default:
        err_dump("unknown type %d for pathname %s", type, pathname);
    }
    return 0;
}

static int compare_file(const char* file1, const char* file2)
{
    FILE* fp1 = fopen(file1, "r");
    FILE* fp2 = fopen(file2, "r");
    if (fp1 == NULL || fp2 == NULL)
    {
        return 0;
    }

    char* buf1 = malloc(4096);
    if (!buf1)
    {
        return 0;
    }

    char* buf2 = malloc(4096);
    if (!buf2)
    {
        return 0;
    }

    int ret = 1;
    while (1)
    {
        int n1 = fread(buf1, 1, 4096, fp1);
        int n2 = fread(buf2, 1, 4096, fp2);
        if (n1 != n2)
        {
            ret = 0;
            break;
        }
        if (n1 == 0)
        {
            break;
        }
        if (memcmp(buf1, buf2, n1) != 0)
        {
            ret = 0;
            break;
        }
    }
    return ret;
}
\end{lstlisting}

\end{spacing}

\end{document}