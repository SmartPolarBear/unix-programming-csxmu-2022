\documentclass[a4paper,twoside]{article}
\usepackage{blindtext}  
\usepackage{geometry}

% Chinese support
\usepackage[UTF8, scheme = plain]{ctex}

% Page margin layout
\geometry{left=2.3cm,right=2cm,top=2.5cm,bottom=2.0cm}


\usepackage{listings}
\usepackage{xcolor}
\usepackage{geometry}
\usepackage{amsmath}
\usepackage{float}
\usepackage{hyperref}

\usepackage{graphics}
\usepackage{graphicx}
\usepackage{epsfig}
\usepackage{float}
\usepackage{wrapfig}

\usepackage{algorithm}
\usepackage[noend]{algpseudocode}

\usepackage{booktabs}
\usepackage{threeparttable}
\usepackage{longtable}
\usepackage{listings}
\usepackage{tikz}
\usepackage{multicol}

\usepackage{caption}
\usepackage{subcaption}

% cite package, to clean up citations in the main text. Do not remove.
\usepackage{cite}

\usepackage{color,xcolor}

%% The amssymb package provides various useful mathematical symbols
\usepackage{amssymb}
%% The amsthm package provides extended theorem environments
\usepackage{amsthm}
\usepackage{amsfonts}
\usepackage{enumerate}
\usepackage{enumitem}
\usepackage{listings}

\usepackage{textcomp}

\usepackage{indentfirst}
\setlength{\parindent}{2em} % Make two letter space in the first paragraph
\usepackage{setspace}
\linespread{1.5} % Line spacing setting
\usepackage{siunitx}
\setlength{\parskip}{0.5em} % Paragraph spacing setting

% \usepackage[contents =22920202204622, scale = 10, color = black, angle = 50, opacity = .10]{background}

\renewcommand{\figurename}{图}
\renewcommand{\lstlistingname}{代码} 
\renewcommand{\tablename}{表格}
\renewcommand{\contentsname}{目录}
\floatname{algorithm}{算法}

\graphicspath{ {images/} }

%%%%%%%%%%%%%
\newcommand{\StudentNumber}{22920202204622}  % Fill your student number here
\newcommand{\StudentName}{熊恪峥}  % Replace your name here
\newcommand{\PaperTitle}{实验(五) 编写程序myfind}  % Change your paper title here
\newcommand{\PaperType}{Unix程序设计} % Replace the type of your report here
\newcommand{\Date}{2022年11月23日}
\newcommand{\College}{信息学院}
\newcommand{\CourseName}{Unix程序设计}
%%%%%%%%%%%%%

%% Page header and footer setting
\usepackage{fancyhdr}
\usepackage{lastpage}
\pagestyle{fancy}
\fancyhf{}
% This requires the document to be twoside
\fancyhead[LO]{\texttt{\StudentName }}
\fancyhead[LE]{\texttt{\StudentNumber}}
\fancyhead[C]{\texttt{\PaperTitle }}
\fancyhead[R]{\texttt{第{\thepage}页,共\pageref*{LastPage}页}}


\title{\PaperTitle}
\author{\StudentName}
\date{\Date}

\lstset{
	basicstyle          =   \sffamily,          % 基本代码风格
	keywordstyle        =   \bfseries,          % 关键字风格
	commentstyle        =   \rmfamily\itshape,  % 注释的风格,斜体
	stringstyle         =   \ttfamily,  % 字符串风格
	flexiblecolumns,                % 别问为什么,加上这个
	numbers             =   left,   % 行号的位置在左边
	showspaces          =   false,  % 是否显示空格,显示了有点乱,所以不现实了
	numberstyle         =   \zihao{-5}\ttfamily,    % 行号的样式,小五号,tt等宽字体
	showstringspaces    =   false,
	captionpos          =   t,      % 这段代码的名字所呈现的位置,t指的是top上面
	frame               =   lrtb,   % 显示边框
}

\lstdefinestyle{PythonStyle}{
	language        =   Python, % 语言选Python
	basicstyle      =   \zihao{-5}\ttfamily,
	numberstyle     =   \zihao{-5}\ttfamily,
	keywordstyle    =   \color{blue},
	keywordstyle    =   [2] \color{teal},
	stringstyle     =   \color{magenta},
	commentstyle    =   \color{red}\ttfamily,
	breaklines      =   true,   % 自动换行,建议不要写太长的行
	columns         =   fixed,  % 如果不加这一句,字间距就不固定,很丑,必须加
	basewidth       =   0.5em,
}

\definecolor{keycolor}{RGB}{172, 42, 42}
\definecolor{mbleu}{RGB}{64,96,127}
\definecolor{vimvert}{RGB}{46, 139, 87}

\lstdefinestyle{MakefileBaseStyle}{
basicstyle=\ttfamily\scriptsize\color{black!90},%
stringstyle=\itshape\color{magenta},%
showstringspaces=false,%
keywordstyle=\bfseries\color{keycolor},%
commentstyle=\color{blue}\slshape,%
framexleftmargin=1mm,%
backgroundcolor=\color{black!2},%
}

\lstdefinestyle{MakefileStyle}{
	otherkeywords={.SUFFIXES},
	morekeywords={SUFFIX, CPP_,},
	moredelim=[is][\color{mbleu}]{/*}{*/},
	style=MakefileBaseStyle,%
	morecomment=[l][commentstyle]{\#},%
	emphstyle={\color{vimvert}},%
	moredelim=[s][\color{vimvert}]{\$(}{)}%
}

\lstdefinestyle{CppStyle}{
	language        =   c++,
	basicstyle      =   \zihao{-5}\ttfamily,
	numberstyle     =   \zihao{-5}\ttfamily,
	keywordstyle    =   \color{blue},
	keywordstyle    =   [2] \color{teal},
	stringstyle     =   \color{magenta},
	commentstyle    =   \color{red}\ttfamily,
	breaklines      =   true,   % 自动换行,建议不要写太长的行
	columns         =   fixed,  % 如果不加这一句,字间距就不固定,很丑,必须加
	basewidth       =   0.5em,
}

\algnewcommand\algorithmicinput{\textbf{Input:}}
\algnewcommand\algorithmicoutput{\textbf{Output:}}
\algnewcommand\Input{\item[\algorithmicinput]}%
\algnewcommand\Output{\item[\algorithmicoutput]}%

\usetikzlibrary{positioning, shapes.geometric}

% 流程图定义基本形状
\tikzstyle{startstop} = [rectangle, rounded corners, minimum width = 2cm, minimum height=1cm,text centered, draw = black]
\tikzstyle{io} = [trapezium, trapezium left angle=70, trapezium right angle=110, minimum width=2cm, minimum height=1cm, text centered, draw=black]
\tikzstyle{process} = [rectangle, minimum width=3cm, minimum height=1cm, text centered, draw=black]
\tikzstyle{decision} = [diamond, aspect = 3, text centered, draw=black]
% 箭头形式
\tikzstyle{arrow} = [->,>=stealth]

\newtheorem{assumption}{Assumption}[section]

\begin{document}
	
%%%%%%%%%%%%%%%%%%%%%%%%%%%%%%%%%%%%%%%%%%%%
\makeatletter % change default title style
\renewcommand*\maketitle{%
	\begin{center} 
		\bfseries  % title 
		{\LARGE \@title \par}  % LARGE typesetting
		\vskip 1em  %  margin 1em
		{\global\let\author\@empty}  % no author information
		{\global\let\date\@empty}  % no date
		\thispagestyle{empty}   %  empty page style
	\end{center}%
	\setcounter{footnote}{0}%
}
\makeatother
%%%%%%%%%%%%%%%%%%%%%%%%%%%%%%%%%%%%%%%%%%%%
	
	
\thispagestyle{empty}

\vspace*{1cm}

\begin{figure}[h]
	\centering
	\includegraphics[width=4.0cm]{logo.png}
\end{figure}

\vspace*{1cm}

\begin{center}
	\Huge{\textbf{\PaperType}}
	
	\Large{\PaperTitle}
\end{center}

\vspace*{1cm}

\begin{table}[h]
	\centering	
	\begin{Large}
		\renewcommand{\arraystretch}{1.5}
		\begin{tabular}{p{3cm} p{5cm}<{\centering}}
			姓\qquad 名 & \StudentName  \\
			\hline
			学\qquad号 & \StudentNumber \\
			\hline
			日\qquad期 & \Date  \\
			\hline
			学\qquad院 & \College  \\
			\hline
			课程名称 & \CourseName  \\
			\hline
		\end{tabular}
	\end{Large}
\end{table}

\newpage

\title{
	\Large{\textcolor{black}{\PaperTitle}}
}
	
	
\maketitle
	
\tableofcontents
 
\newpage
\setcounter{page}{1}

\begin{spacing}{1.2}

\section{实验内容}

编写程序myfind

\subsection{命令语法}
myfind  \textlangle pathname\textrangle \ [-comp \textlangle filename\textrangle \ \textbar \ -name \textlangle str\textrangle  …]

\subsection{命令语义}

\begin{enumerate}
	\item myfind  \textlangle pathname\textrangle 的功能:除了具有与程序4-7相同的功能外,还要输出在\textlangle pathname\textrangle  目录子树之下,文件长度不小于4096字节的常规文件的数量。程序不允许打印出任何路径名。
\end{enumerate}

\section{程序设计与实现}

\subsection{程序设计}

\begin{wrapfigure}{r}{0.5\textwidth}
	\centering
	\caption{流程图}
	\label{fig:flowchart}
	\begin{tikzpicture}[node distance=1cm]
		%定义流程图具体形状
		\node[startstop](start){进入回调函数};
		\node[decision, below of = start, yshift = -1cm](dec1){参数3是否为$FTW\_F$?};
		\node[process, below of = dec1, yshift = -1cm](write){执行相应查找功能};
		\node[io, below of = write, yshift = -1cm](out1){输出路径};
		\node[startstop, below of = out1, yshift = -1cm](stop){结束};
		\coordinate (point1) at (-3cm, -2cm);
		%连接具体形状
		\draw [arrow] (start) -- (dec1);
		\draw (dec1) -- node [above] {N} (point1);
		\draw [arrow] (point1) |- (stop);
		\draw [arrow] (write) -- (out1);
		\draw [arrow] (dec1) -- node [right] {Y} (write);
		\draw [arrow] (out1) -- (stop);

	\end{tikzpicture}
\end{wrapfigure}


首先简单地分析课本程序4-7。$myftw$接受一个路径和一个回调函数作为参数。它为路径分配内存,
然后调用$dopath$完成路径遍历的实际功能。$dopath$中通过$lstat$获取文件的信息,然后调用
回调函数。其中调用的第三个参数将遍历到的文件系统节点分为了文件、目录、不可读的目录、不能执行$stat$的
文件4类。然后$dopath$对可读的目录进行递归调用。这样实现了对目录树的先序遍历。

由于$myftw$提供了基本可用的接口来在遍历目录树的时候进行自定义的操作,为了完成实验要求,就需要对应的回调函数。
只需要微调4-7中的$myfunc$,当文件尺寸比4096大时进行计数,就能完成所需要的功能。


\subsection{程序实现}

代码见\nameref{sec:code}中的代码~\ref{code:impl}。任务1的实现较为简单,只需要在回调函数的参数3为FTW\_F且文件类型是S\_IFREG时,进一步
判断文件的大小是否大于4096即可。


\section{程序测试}

运行结果如图~\ref{fig:results}。该结果和目录中文件的实际大小相符。可见实现是正确的。

\begin{figure}[htb]
	\centering
	\caption{运行结果}
	\label{fig:results}
	\includegraphics[width=0.6\textwidth]{result.png}
\end{figure}


\appendix
\clearpage
\section*{附录:代码清单}
\addcontentsline{toc}{part}{附录:代码清单}
\label{sec:code}

\subsection{Makefile}

\begin{lstlisting}[numbers=left,style=MakefileStyle,caption=Makefile,label={code:makefile}]
OBJS = main.o \
	   error.o \
	   pathalloc.o \

CC = gcc
CFLAGS = -Wall -g -std=c99

%.o: %.c
	$(CC) $(CFLAGS) -c $< -o $@

myfind: $(OBJS)
	$(CC) $(CFLAGS) $(OBJS) -o myfind

all: myfind

clean:
	rm -f *.o myfind

.PHONY: clean all
\end{lstlisting}

\subsection{myfind.c}

\begin{lstlisting}[numbers=left,style=CppStyle,caption=程序实现,label={code:impl}]
#include "apue.h"
#include <dirent.h>
#include <limits.h>

#include <libgen.h> // basename

typedef int (Callback)(const char *, const struct stat *, int);

static Callback simple_statistic;

static int myftw(char *, Callback *);

static int dopath(Callback *);

static long nreg, ndir, nblk, nchr, nfifo, nslink, nsock, ntot, nless4k;

#define NARGS 16

int main(int argc, char *argv[])
{
	int ret;
	if (!(argc == 2))
	{
		err_quit("usage: myfind <params>");
	}

	if (argc == 2)
	{
		ret = myftw(argv[1], simple_statistic);
		ntot = nreg + ndir + nblk + nchr + nfifo + nslink + nsock;
		if (ntot == 0)
		{
			ntot = 1;
		}

		printf("%ld", nless4k);
	}

	return ret;
}

#define FTW_F 1
#define FTW_D 2
#define FTW_DNR 3
#define FTW_NS 4

static char *fullpath;
static size_t pathlen;

static int myftw(char *pathname, Callback *func)
{
	fullpath = path_alloc(&pathlen);

	if (pathlen <= strlen(pathname))
	{
		pathlen = strlen(pathname) * 2;
		if ((fullpath = realloc(fullpath, pathlen)) == NULL)
		{
			err_sys("realloc failed");
		}
	}
	strcpy(fullpath, pathname);
	return (dopath(func));
}

static int dopath(Callback *func)
{
	struct stat statbuf;
	struct dirent *dirp;
	DIR *dp;
	int ret, n;
	if (lstat(fullpath, &statbuf) < 0)
	{
		return (func(fullpath, &statbuf, FTW_NS));
	}
	if (S_ISDIR(statbuf.st_mode) == 0)
	{
		return (func(fullpath, &statbuf, FTW_F));
	}

	if ((ret = func(fullpath, &statbuf, FTW_D)) != 0)
	{
		return (ret);
	}
	n = strlen(fullpath);
	if (n + NAME_MAX + 2 > pathlen)
	{
		pathlen *= 2;
		if ((fullpath = realloc(fullpath, pathlen)) == NULL)
		{
			err_sys("realloc failed");
		}
	}
	fullpath[n++] = '/';
	fullpath[n] = 0;
	if ((dp = opendir(fullpath)) == NULL)
	{
		return (func(fullpath, &statbuf, FTW_DNR));
	}
	while ((dirp = readdir(dp)) != NULL)
	{
		if (strcmp(dirp->d_name, ".") == 0 || strcmp(dirp->d_name, "..") == 0)
		{
			continue;
		}
		strcpy(&fullpath[n], dirp->d_name);
		if ((ret = dopath(func)) != 0)
		{
			break;
		}
	}
	fullpath[n - 1] = 0;
	if (closedir(dp) < 0)
	{
		err_ret("can't close directory %s", fullpath);
	}
	return (ret);
}

static int simple_statistic(const char *pathname, const struct stat *statptr, int type)
{
	switch (type)
	{
	case FTW_F:
		switch (statptr->st_mode & S_IFMT)
		{
		case S_IFREG:
			nreg++;
			if (statptr->st_size > 4096)
			{
				nless4k++;
			}
			break;
		case S_IFBLK:
			nblk++;
			break;
		case S_IFCHR:
			nchr++;
			break;
		case S_IFIFO:
			nfifo++;
			break;
		case S_IFLNK:
			nslink++;
			break;
		case S_IFSOCK:
			nsock++;
			break;
		case S_IFDIR:
			err_dump("for S_IFDIR for %s", pathname);
		}
		break;
	case FTW_D:
		ndir++;
		break;
	case FTW_DNR:
		err_ret("can't read directory %s", pathname);
		break;
	case FTW_NS:
		err_ret("stat error for %s", pathname);
		break;
	default:
		err_dump("unknown type %d for pathname %s", type, pathname);
	}
	return 0;
}
\end{lstlisting}

\end{spacing}

\end{document}